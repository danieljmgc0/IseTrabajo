\documentclass[a4paper,12pt]{report}
\usepackage[utf8]{inputenc}
\usepackage{graphicx}
\usepackage{geometry}
\usepackage[spanish,es-tabla]{babel}
\usepackage[utf8]{inputenc}
\usepackage[T1]{fontenc}
\geometry{margin=3cm}

\begin{document}

\begin{titlepage}
    \centering
    \vspace*{3cm}
    {\Huge\bfseries Trabajo 1: Diseño de un experimento controlado \par}
    \vspace{2cm}
    {\Large\itshape Daniel Jiménez García \par}
    {\Large\itshape Sergio Muñoz Gómez \par}
    {\Large\itshape Ángela Caiqing Pousada Morán \par}
    \vspace{1cm}
    {\Large\itshape Fecha: \today \par}
    \vfill
    {\large MITTS \par}
    {\large UPV \par}
\end{titlepage}





\chapter{Motivación}

\section{Problema a investigar}

\section{Definición del experimento}

\section{Contexto}

\section{Implicaciones e impacto}


\chapter{Trabajos relacionados}

\chapter{Descripción del diseño}

\section{Hipótesis y variables}

\section{Diseño del experimento}

\section{Selección de sujetos}

\section{Objetos e instrumentación}

\section{Evaluación de la validez}





\section{La primera parte es...}
Resumen. Ya al final
Motivación. Definición del expermiento. COntexto. 

Definición: Queremos evaluar 2 metodologías de evaluación de usuarios con respecto a la usabilidad de un sistema informático. La idea es comparar el A/B testing con el Recorrido Pluraista (entrevistas a usuarios sobre su trabajo con prototipos)

Varibales de interes
DEtectar defectos, no falsos positivos. Tipos de defectos en métodos de inspección 

Efectividad: porcentaje de acierto, tareas completadas
Eficiencia:
Satisfacción:

EN inspección detectar el mayor número de errores en los usuarios para corregirlo

Se puede medir el esfuerzo cognitivo del usuario
Comparar 2 sistemas en vez de uno para no ser dependiente de dicho sistemas, pero ambos deben tener una complejidad parecida y entonces condiciona la evaluación:wq


COntexto: 





\end{document}
